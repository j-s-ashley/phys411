\documentclass[12 pt]{article}        	%sets the font to 12 pt and says this is an article (as opposed to book or other documents)
\usepackage{amsfonts, amssymb}					% packages to get the fonts, symbols used in most math
  
%\usepackage{setspace}               		% Together with \doublespacing below allows for doublespacing of the document

\oddsidemargin=-0.5cm                 	% These three commands create the margins required for class
\setlength{\textwidth}{6.5in}         	%
\addtolength{\voffset}{-20pt}        		%
\addtolength{\headsep}{25pt}           	%



\pagestyle{myheadings}                           	% tells LaTeX to allow you to enter information in the heading
\markright{Murphy Waggoner\hfill \today \hfill} 	% put your name instead of Murphy Waggoner 
																									% and put the proposition number from the book
                                                	% LaTeX will put your name on the left, the date the paper 
                                                	% is generated in the middle 
                                                 	% and a page number on the right



\newcommand{\eqn}[0]{\begin{array}{rcl}}%begin an aligned equation - allows for aligning = or inequalities.  Always use with $$ $$
\newcommand{\eqnend}[0]{\end{array} }  	%end the aligned equation

\newcommand{\qed}[0]{$\square$}        	% make an unfilled square the default for ending a proof

%\doublespacing                         	% Together with the package setspace above allows for doublespacing of the document

\begin{document}												% end of preamble and beginning of text that will be printed
Given:

( \mathbf{r}_1 ) and ( \mathbf{r}_2 ) are the positions of particles 1 and 2, respectively.
( \mathbf{R} ) is the center of mass (CM) position.
( \mathbf{r} = \mathbf{r}_1 - \mathbf{r}_2 ) is the relative position vector.
( m_1 ) and ( m_2 ) are the masses of particles 1 and 2, respectively.
( M = m_1 + m_2 ) is the total mass.
The center of mass position ( \mathbf{R} ) is given by: [ \mathbf{R} = \frac{m_1 \mathbf{r}_1 + m_2 \mathbf{r}_2}{M} ]

We need to express ( \mathbf{r}_1 ) and ( \mathbf{r}_2 ) in terms of ( \mathbf{R} ) and ( \mathbf{r} ).

First, solve for ( \mathbf{r}_1 ) and ( \mathbf{r}_2 ) in terms of ( \mathbf{R} ) and ( \mathbf{r} ):

From the definition of ( \mathbf{r} ): [ \mathbf{r} = \mathbf{r}_1 - \mathbf{r}_2 ] [ \mathbf{r}_1 = \mathbf{r} + \mathbf{r}_2 ]
Substitute ( \mathbf{r}_1 ) into the center of mass equation: [ \mathbf{R} = \frac{m_1 (\mathbf{r} + \mathbf{r}_2) + m_2 \mathbf{r}_2}{M} ] [ \mathbf{R} = \frac{m_1 \mathbf{r} + m_1 \mathbf{r}_2 + m_2 \mathbf{r}_2}{M} ] [ \mathbf{R} = \frac{m_1 \mathbf{r} + (m_1 + m_2) \mathbf{r}_2}{M} ] [ \mathbf{R} = \frac{m_1 \mathbf{r} + M \mathbf{r}_2}{M} ] [ \mathbf{R} = \frac{m_1 \mathbf{r}}{M} + \mathbf{r}_2 ]
Solve for ( \mathbf{r}_2 ): [ \mathbf{r}_2 = \mathbf{R} - \frac{m_1 \mathbf{r}}{M} ]
Substitute ( \mathbf{r}_2 ) back into ( \mathbf{r}_1 = \mathbf{r} + \mathbf{r}_2 ): [ \mathbf{r}_1 = \mathbf{r} + \left( \mathbf{R} - \frac{m_1 \mathbf{r}}{M} \right) ] [ \mathbf{r}_1 = \mathbf{R} + \mathbf{r} - \frac{m_1 \mathbf{r}}{M} ] [ \mathbf{r}_1 = \mathbf{R} + \left( 1 - \frac{m_1}{M} \right) \mathbf{r} ] [ \mathbf{r}_1 = \mathbf{R} + \frac{m_2 \mathbf{r}}{M} ]
Thus, we have: [ \mathbf{r}_1 = \mathbf{R} + \frac{m_2 \mathbf{r}}{M} ] [ \mathbf{r}_2 = \mathbf{R} - \frac{m_1 \mathbf{r}}{M} ]

These equations verify that the positions of the two particles can indeed be written in terms of the center of mass and relative positions as given.

Chat with Copilot





\end{document}